\documentclass[]{book}
\usepackage{lmodern}
\usepackage{amssymb,amsmath}
\usepackage{ifxetex,ifluatex}
\usepackage{fixltx2e} % provides \textsubscript
\ifnum 0\ifxetex 1\fi\ifluatex 1\fi=0 % if pdftex
  \usepackage[T1]{fontenc}
  \usepackage[utf8]{inputenc}
\else % if luatex or xelatex
  \ifxetex
    \usepackage{mathspec}
  \else
    \usepackage{fontspec}
  \fi
  \defaultfontfeatures{Ligatures=TeX,Scale=MatchLowercase}
\fi
% use upquote if available, for straight quotes in verbatim environments
\IfFileExists{upquote.sty}{\usepackage{upquote}}{}
% use microtype if available
\IfFileExists{microtype.sty}{%
\usepackage{microtype}
\UseMicrotypeSet[protrusion]{basicmath} % disable protrusion for tt fonts
}{}
\usepackage{hyperref}
\hypersetup{unicode=true,
            pdftitle={Tayside Acute Care Guide},
            pdfborder={0 0 0},
            breaklinks=true}
\urlstyle{same}  % don't use monospace font for urls
\usepackage{natbib}
\bibliographystyle{apalike}
\usepackage{longtable,booktabs}
\usepackage{graphicx,grffile}
\makeatletter
\def\maxwidth{\ifdim\Gin@nat@width>\linewidth\linewidth\else\Gin@nat@width\fi}
\def\maxheight{\ifdim\Gin@nat@height>\textheight\textheight\else\Gin@nat@height\fi}
\makeatother
% Scale images if necessary, so that they will not overflow the page
% margins by default, and it is still possible to overwrite the defaults
% using explicit options in \includegraphics[width, height, ...]{}
\setkeys{Gin}{width=\maxwidth,height=\maxheight,keepaspectratio}
\IfFileExists{parskip.sty}{%
\usepackage{parskip}
}{% else
\setlength{\parindent}{0pt}
\setlength{\parskip}{6pt plus 2pt minus 1pt}
}
\setlength{\emergencystretch}{3em}  % prevent overfull lines
\providecommand{\tightlist}{%
  \setlength{\itemsep}{0pt}\setlength{\parskip}{0pt}}
\setcounter{secnumdepth}{5}
% Redefines (sub)paragraphs to behave more like sections
\ifx\paragraph\undefined\else
\let\oldparagraph\paragraph
\renewcommand{\paragraph}[1]{\oldparagraph{#1}\mbox{}}
\fi
\ifx\subparagraph\undefined\else
\let\oldsubparagraph\subparagraph
\renewcommand{\subparagraph}[1]{\oldsubparagraph{#1}\mbox{}}
\fi

%%% Use protect on footnotes to avoid problems with footnotes in titles
\let\rmarkdownfootnote\footnote%
\def\footnote{\protect\rmarkdownfootnote}

%%% Change title format to be more compact
\usepackage{titling}

% Create subtitle command for use in maketitle
\providecommand{\subtitle}[1]{
  \posttitle{
    \begin{center}\large#1\end{center}
    }
}

\setlength{\droptitle}{-2em}

  \title{Tayside Acute Care Guide}
    \pretitle{\vspace{\droptitle}\centering\huge}
  \posttitle{\par}
    \author{}
    \preauthor{}\postauthor{}
    \date{}
    \predate{}\postdate{}
  
\usepackage{booktabs}

\begin{document}
\maketitle

{
\setcounter{tocdepth}{1}
\tableofcontents
}
\hypertarget{section}{%
\chapter*{}\label{section}}
\addcontentsline{toc}{chapter}{}

\includegraphics[width=1\linewidth]{banner-acute-care-guide}

BY PAUL FETTES

This guide is aimed at medical students and junior doctors to enable them to know what to do in an acute situation. It is the culmination of a lot of work by a lot of people. It is designed for use in Tayside and has many specific protocols and contact numbers for that reason, but many of the approaches described are generic and it will have much wider application.

Please use this guide with care and a healthy dose of common sense. It should not be relied upon to replace clinical knowlege or experience and the first thing you should do in many of the situations is call for senior help.

Your feedback is vital to the development of this guide.
Please contact me via the feedback page if you have any feedback, if links don't work or are out of date, if anything is missing, or if you can think of a way of improving it.

\hypertarget{resuscitation}{%
\chapter{Resuscitation}\label{resuscitation}}

\hypertarget{advanced-life-support}{%
\section{Advanced Life Support}\label{advanced-life-support}}

In-hospital resuscitation
Adult Advanced Life Support
Tachycardia
Bradycardia
Paediatric Advanced Life Support
Newborn Life Support

`Resuscitation` can be divided into 2 categories:

Peri-Arrest Patients
Patients in Cardio-Respiratory Arrest
It is very difficult to give one resuscitation plan which can apply to every patient. Each patient will require a tailored treatment plan for their acute insult. Broadly speaking, each critically unwell patient will need basic assessment and investigations to be performed in a bid to gain a diagnosis / differential list while a basic resuscitation is initiated and subsequently tailored as clinical / laboratory information becomes available.

Basic Principles (apply to both categories):

GET HELP IMMEDIATELY -- Medical / Nursing / HAN etc

Resuscitation of a patient requires a co-ordinated team response

Follow ABCDE Approach and FREQUENTLY REASSES

A Team Leader should co-ordinate events (usually most senior member of staff)
Gather information -- Patient (if able), Notes, Charts, Allergy Bands, Staff on Ward
Communication as a Team is VITAL
During the initial phase of resuscitation, a co-ordinated assessment of ABCDE must occur efficiently with each member of the team appointed to a system(s) or tasks. Often assessments and interventions can be carried synchronously with one person assessing and another attaching monitoring, cannulating and taking bloods etc.

AssessmentInterventions Interventions
Clinical
Monitoring
Airway Patent / Talking
Additional Noises (wheeze / stridor) Airway Positioning (?Link)
Adjuncts
(Nasopharyngeal or Guedel)
Is Definitive Airway Required?
Breathing Trachea
Inspection
Palpation
Percussion
Auscultation Saturations Probe
Respiratory Rate HIGH FLOW OXYGEN
ABG (?Link)
Consider :
Nebulisers
Needle Thoracocentesis
Chest Drain
CPAP / BiPAP
Circulation Pulse
BP
Skin colour / Temperature
Capillary Refill
Heart Sounds Cardiac Monitor
BP (NIBP / Arterial)
ECG 2 x LARGE IV ACCESS
Blood samples for Lab
(See Table in Link)
FLUID RESUSCIATION with 500ml boluses of NaCl 0.9\%, assessing response each time
Disability AVPU or GCS
(See Table in Link)
Pupils -- Size and Reaction
Posture
Peripheral Neurology TREATMENT OF ABNORMAL PARAMETERS IN ABC MAY IMPROVE D
Exposure Temperature
BM
External Signs of Injury
Environmental Clues Tympanic / Core Temperature
BM Stick +/- Ketone Stick Antibiotics
Glucose Bolus / Glucagon
REMEMBER TO CONTINUALY REASSES ABCDE AFTER EACH INTERVENTION OR IF ANYTHING CHANGES

Following the basic principles above will allow rapid assessment and stabilisation of the critically ill patient whilst examination and investigations give you an educated differential diagnosis list. There are many acute medical and surgical presentations which shall be covered in the coming chapters.

With all critically unwell patients, consideration of escalation of level of care should be considered. This could be to Coronary Care, High Dependency Unit (Medical or Surgical) or even Intensive Care. The decision to escalate should be made by the most senior members of the team present, bearing in mind that a ceiling of treatment may need to be considered.

The following criteria may guide when it is appropriate to consider referral to the Intensive Care Unit:

Links
SEWS
Shock
Massive Transfusion
Code Red
AVPU or GCS

AVPU
A -- Alert and Communicating
V -- Responds only to Voice
P -- Responds only to Pain
U -- Unresponsive

GCS
Score MINIMUM 3 to MAXIMUM 15

E 4 Eyes open spontaneously
E 3 Eyes open to Voice
E 2 Eyes open to Pain
E 1 No opening of the eyes

M 6 Obeys Commands
M 5 Localises to Stimulus
M 4 Flexes to Painful Stimulus
M 3 Abnormal Flexion to Pain
M 2 Extends to Painful Stimulus
M 1 No Motor Response

V 5 Orientated Speech
V 4 Confused content of Speech
V 3 Inappropriate Speech
V 2 Incomprehensible Speech
V 1 No Vocal Response

Even programme where tick boxes and generates number

Blood Samples for Lab
Selected from ICE menu -- Think about:

FBC, Co-agulation Screen / INR
U+E's, Calcium, Magnesium, LFT's, CRP, Amylase, Troponin
Lactate, Glucose
Group and Save, Cross Match
Blood Cultures -- Peripheral and Central

ATOMFC
Acronym for Life-Threatening Airway Insults

A -- Airway Loss
T -- Tension Pneumothorax
O -- Open Pneumothorax
M -- Massive Haemothorax
F -- Flail Chest
C -- Cardiac Tamponade

Advanced Life Support
Code Red
Major Incident / Mass Casualty Incident

\hypertarget{major-incident}{%
\section{Major Incident}\label{major-incident}}

\hypertarget{haemodynamically-unstable-polytrauma-code-red}{%
\section{Haemodynamically Unstable Polytrauma (Code Red)}\label{haemodynamically-unstable-polytrauma-code-red}}

\hypertarget{adult-massive-haemorrhage-policy}{%
\section{Adult Massive Haemorrhage Policy}\label{adult-massive-haemorrhage-policy}}

\hypertarget{reversing-oral-anticoagulants-in-massive-haemorrhage}{%
\section{Reversing oral anticoagulants in massive haemorrhage}\label{reversing-oral-anticoagulants-in-massive-haemorrhage}}

\hypertarget{airprob}{%
\chapter{Airway Problems}\label{airprob}}

\hypertarget{emergency-contacts}{%
\section{Emergency Contacts}\label{emergency-contacts}}

\hypertarget{airway-assessment}{%
\section{Airway Assessment}\label{airway-assessment}}

\hypertarget{threaten-the-airway}{%
\section{Threaten the Airway}\label{threaten-the-airway}}

\hypertarget{ent-emergencies}{%
\section{ENT Emergencies}\label{ent-emergencies}}

\hypertarget{breath}{%
\chapter{Breathing}\label{breath}}

\hypertarget{introduction}{%
\section{Introduction}\label{introduction}}

\hypertarget{useful-contacts}{%
\section{Useful Contacts}\label{useful-contacts}}

\hypertarget{respiratory-emergencies}{%
\section{Respiratory Emergencies}\label{respiratory-emergencies}}

\hypertarget{non-invasive-ventilation-niv}{%
\section{Non-invasive Ventilation (NIV)}\label{non-invasive-ventilation-niv}}

\hypertarget{how-to-set-up-niv-extracted-from-bts-guideline}{%
\section{How to set up NIV (extracted from BTS guideline)}\label{how-to-set-up-niv-extracted-from-bts-guideline}}

\hypertarget{tuberculosis}{%
\section{Tuberculosis}\label{tuberculosis}}

\hypertarget{pneumothorax}{%
\section{Pneumothorax}\label{pneumothorax}}

\hypertarget{haemopneumothorax-haemothorax}{%
\section{Haemopneumothorax (haemothorax)}\label{haemopneumothorax-haemothorax}}

\hypertarget{non-respiratory-causes-of-respiratory-compensation}{%
\section{Non Respiratory causes of respiratory compensation}\label{non-respiratory-causes-of-respiratory-compensation}}

\hypertarget{assessment-of-breathing}{%
\section{Assessment of Breathing}\label{assessment-of-breathing}}

\hypertarget{cardiovasc}{%
\chapter{Cardiology and Vascular}\label{cardiovasc}}

Cardiology and Vascular
Cardiology
Acute pulmonary oedema
Artrial Fibrillation Management
Management of device patients
Optimal Reperfusion Therapy (AMU)
Optimal Reperfusion Therapy (PRI)
Shock and Sepsis
Emergency Contact Numbers
Definition of shock
General approach to shock
Cardiogenic Shock
Severe Sepsis \& Septic Shock
Hypovolaemic Shock
Anaphylactic Shock
SEPSIS Six Bundle
Antibiotic Man
Vascular
Acute Limb Ischaemia
Acute Thoracic Aortic Dissection
Ruptured Abdominal Aortic Aneurysm
Vascular Antibiotic Policy
Adult Massive Haemorrhage Policy
Reversing oral anticoagulants in massive haemorrhage
Fresh Frozen Plasma (FFP) Cryo Guideline
Platelet Transfusion Guideline

\hypertarget{neuromfe}{%
\chapter{Neurology / MFE}\label{neuromfe}}

Meningitis
Management of Acute Stroke
Management of Epileptic Seizure
Suspected Spinal Cord Compression
Seizures in a Known Epileptic
Delirium
Dysphagia

\hypertarget{renal-endocrine}{%
\chapter{Renal \& Endocrine}\label{renal-endocrine}}

Review Medication
Urinary tract infection
Metabolic acidosis
Recognition of AKI
AKI Guidelines
Management of contrast-induced nephropathy
DKA
HHS
Hypoglycaemia
Hyperkalaemia
Hypokalaemia
Hyponatraemia
Hypocalcaemia
Addisonian crisis
Thyrotoxicosis
Phaeochromocytoma

\hypertarget{obsandgynae}{%
\chapter{Obstetrics \& Gynaecological}\label{obsandgynae}}

Obstetric Emergency Contacts
Ante-partum Haemorrhage (APH)
Post Partum Haemorrhage (PPH)
Severe Preeclampsia / Eclampsia
Sepsis in Obstetrics
Shoulder Dystocia
Maternal Collapse
Cord Prolapse
Breech in Labour
Uterine Inversion

\hypertarget{psychemerg}{%
\chapter{Psychiatric Emergencies}\label{psychemerg}}

Introduction
Contacts
Psychiatric Assessment
Common Presentations
Deliberate self-harm / Attempted Suicide
Schizophrenia
Depression
Drugs and Alcohol
Delirium
Mental Health Act
Additional Information

\hypertarget{surg}{%
\chapter{Surgical}\label{surg}}

General / Trauma Surgery by Justyna Szczachor \& Aaron Quyn

Acute Pancreatitis
Assessment of severity
Management of gallstones in gallstone pancreatitis
Abdominal trauma
Blunt abdominal trauma
Penetrating abdominal trauma
Thoracic trauma
Rib fractures
Peritonitis
Significant Lower GI Bleed
Massive Acute Rectal Bleeding
Vascular Surgery by Stuart Suttie
ENT Surgery by Stephen Jones
Contacts
Epistaxis
Orbital Cellulitis
Throat
Sore Throat
Quinsy
Glandular Fever
Epiglottitis / Supraglottitis
Stridor
Post-Tonsillectomy Haemorrhage

\hypertarget{pain}{%
\chapter{Pain}\label{pain}}

by Mark Henderson

Pain
Key Contacts
Background Information
Types of pain
How to assess a patient in pain
Management of acute pain
Common analgesics for use in adult patients
Non-opioid analgesics
Common opioid analgesics
Opioid titration and Post-operative nausea guideline
Guidelines for opioid use
Opioid conversion
Approximate opioid equivalence chart
Managing opioid side effects
Opioid toxicity
Excessive sedation and respiratory arrest
Itch
Constipation
Other analgesics and adjuvant therapies
Management of neuropathic pain
Patient Controlled Analgesia
Setting up a PCA
PCA Prescription
PCA Pit-falls
PCA Troubleshooting
PCA Troubleshooting Algorithm
PCA Morphine step down
Epidural analgesia
Background information
Complications
Common problems
Hypotension
Inadequate analgesia
High sensory block
Dense motor block
Leaking epidural
Epidural disconnection
Epidural falling out
Urinary retention
Evolution of other medical problems
How to assess a patient with an epidural
Resources and further reading

\hypertarget{death}{%
\chapter{Death}\label{death}}

by Justyna Szczachor, Iain Kennedy \& Karen Pearson

Useful Contacts
Deaths of Patients
Confirming Death
Breaking Bad News
Documenting Death
Coping with death

\hypertarget{poisions}{%
\chapter{Poisons}\label{poisions}}

by Gareth Patton

TOXBASE

Antidotes
Background
Decontamination
Elimination
Management
Resources
Summary

\hypertarget{majinc}{%
\chapter{Major Incident}\label{majinc}}

Haemodynamically Unstable Polytrauma (Code Red)
Patient process
Massive Transfusion/Blood Loss Algorithm
Adult Massive Haemorrhage Policy
Platelet Transfusion Guideline
Reversing oral anticoagulants in massive haemorrhage
Fresh Frozen Plasma (FFP) Cryo Guideline

\hypertarget{practproc}{%
\chapter{Practical Procedures}\label{practproc}}

by Karen Pearson

General Tips
Access \& invasive monitoring
Difficult Access (including Femoral Stab)
Arterial line insertion
Central line insertion
Samples for Investigation
Pleural tap/needle thoracocentesis
Ascitic tap/drain insertion
Lumbar puntcure
Respiratory
Chest drain insertion
Other
Urethral Catheterisation
Nasogastric tube insertion
Resources

\hypertarget{calc}{%
\chapter{Calculations}\label{calc}}

\hypertarget{paeds}{%
\chapter{Paediatrics}\label{paeds}}

Emergency Contacts
Anaesthetic and Transfer
Cardiovascular
Cardiac Arrest
Arrhythmia
Bradycardia Algorithm
SVT Algorithm
Ventricular Tachycardia Algorithm
Burns Management
Fluid Resuscitation
Paediatric Endocrine
Adrenal Crises
DKA
Hypoglycaemia
Infectious Diseases
Meningococcal Disease Management
Paediatric Lumbar Puncture
Paediatric Neurology
Status Epilepticus Algorithm
Opthalmology
Orbital Cellulitis Management
Respiratory
Acute Athsma Exacerbation Algorithm
Anaphlaxis Algorithm
Bronchiolitis Management
Upper Airway Obstruction Algorithm
Viral Croup Management

\hypertarget{infection}{%
\chapter{Infection}\label{infection}}

Tayside Formulary

Antibiotic Man

\hypertarget{feedback}{%
\chapter{Feedback}\label{feedback}}

Your feedback is vital to the development of this guide. Please contact us if links don't work or are out of date, if anything is missing and needs to be added, or if you can think of any other way of improving it below. Thank you.

\hypertarget{credits}{%
\chapter{Credits}\label{credits}}

Chris Kennedy, TILT setup platform and website.

Andrew Melvin has contributed to the editing, design and content of this site.

Kirsty McNeil has provided highly imaginative illustrations for many of the sections.

Jessica Thompson did the wonderful drawings of aortic dissection in the vascular surgery section.

Christopher McCann developed App on iOS and Android.

\bibliography{book.bib,packages.bib}


\end{document}
